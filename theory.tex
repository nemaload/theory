\documentclass[letterpaper,10pt,twoside]{article}

\usepackage[utf8]{inputenc}
\usepackage[T1]{fontenc}
\usepackage[american]{babel}

%\usepackage{datetime}
%\newdateformat{ymd}{\THEYEAR/\twodigit{\THEMONTH}/\twodigit{\THEDAY}}

\usepackage{color}
%\usepackage{verbatim}

%\usepackage{soul}
\usepackage{relsize}

\usepackage{isomath}
\newcommand{\mat}[1]{\matrixsym{#1}}
\renewcommand{\vec}[1]{\vectorsym{#1}}
\usepackage{titlesec}
\titleformat*{\section}{\Large\bfseries\scshape}
\titleformat*{\subsection}{\large\bfseries\scshape}
\titleformat*{\subsubsection}{\bfseries\scshape}
\titlespacing*{\section}{0pt}{3.5mm plus1mm minus0.5mm}{1.7mm plus0.4mm minus0.4mm}
\titlespacing*{\subsection}{0pt}{2.4mm plus.5mm minus0.2mm}{0.9mm plus0.2mm minus0.2mm}
\titlespacing*{\subsubsection}{0pt}{1.3mm plus.6mm minus0.2mm}{0.1mm plus0.1mm minus0.1mm}
\titleformat{\paragraph}[runin]{\bfseries}{}{0pt}{}{}
\titlespacing{\paragraph}{0pt}{0.4mm plus0.5mm minus0.3mm}{1.9em plus0.6em minus0.8em}
\usepackage{caption}
\usepackage{subfig}
\DeclareCaptionFormat{ruled}{\hrulefill\\#1#2#3}
\captionsetup[figure]{position=top,font={up,small},labelfont={up,bf},margin=8mm,labelsep=period,justification=centerlast,format=ruled}
%\usepackage[subfigure]{ccaption}
%\captionnamefont{\bfseries}
%\captiontitlefont{\small}
%\captiondelim{. }
%\captionstyle{\centerlastline}
%\precaption{\rule{\linewidth}{0.4pt}\par}
%\postcaption{\vspace{1mm}}
\setlength{\abovecaptionskip}{6mm plus3mm}
\setlength{\belowcaptionskip}{2mm plus3mm}
\usepackage[alwaysadjust]{paralist}
\setdefaultleftmargin{1.1em}{1.5em}{0.8ex}{0.8ex}{0.8ex}{0.8ex}
\usepackage{parskip}
\usepackage{graphicx}
%\usepackage{framed}
\usepackage{array}
\usepackage{booktabs}
%\usepackage{tabularx}
%\usepackage{ltxtable}
%\usepackage{multicol}
\usepackage{tabu}
\usepackage{fancyhdr}
\usepackage{titling}
\usepackage{pdflscape}
\usepackage{authblk}
\usepackage[overload]{ragged2e}
\usepackage[section,above]{place ins}

%title configuration
\setlength{\droptitle}{-1cm}
\pretitle{\begin{center}\LARGE}
\posttitle{\end{center}}

%author block hack
\makeatletter
\newsavebox{\affilsym}
\newcommand\lift[1]{{\fontsize{8}{8}\selectfont\vtop{\vskip -2.78ex \hbox{\hspace{0.14em}#1}}}}
\renewcommand\AB@authnote[1]{\lift{#1}}
\renewcommand\AB@affilnote[1]{\normalfont\textbf{#1}\hspace*{0.4em plus 0.6em}}
\renewcommand\AB@affilsep{\protect\par\protect\vspace{0.1pt}\protect\justify\protect\Affilfont}
\renewcommand\AB@affilsepx{\protect\hskip 1em plus 0.4em \protect\Affilfont}
\makeatother
\renewcommand\Authsep{\hskip 0.39em minus 0.05em}
\renewcommand\Authands{~{\fontsize{8.7}{8.4}\selectfont{and}}\hskip 0.51em minus 0.08em}
\renewcommand\Authfont{\fontsize{9}{12}\selectfont\setlength{\baselineskip}{15pt}}
\renewcommand\Affilfont{\footnotesize}
\protected\def\jointfirst{\vartriangleleft}
\newcommand{\jointlast}{\hspace{-0.3pt}\vartriangleright}

%date configuration
\predate{\begin{center}\small({\color{red}Draft} produced }
\postdate{)\end{center}}

%page numbering
\fancypagestyle{plain}{
\fancyhead{}
\fancyhead[LE,RO]{\thepage}
\fancyfoot{}
}

\usepackage{abstract}
\setlength{\absleftindent}{3.7em}
\setlength{\absrightindent}{3.7em}

%\usepackage{marvosym}
\usepackage{amsmath}
\usepackage{amsthm}
\usepackage{mathtools}
%\usepackage{cancel}
\usepackage[binary-units=true]{siunitx}
\DeclareSIUnit\Molar{\textsc{m}}
\newcommand{\ca}{{\sim}}
\sisetup{input-protect-tokens=\ca\approx\sim\dots\gg\ll\pm,input-symbols=\ca,input-ignore={,},input-decimal-markers={.},range-phrase=\textrm{--},range-units=single,product-units=repeat,retain-unity-mantissa=false,per-mode=symbol,list-units=single}

\newcommand\sub[1]{_{\text{\scriptsize #1}}}
\renewcommand\d{\text{d}}
\newcommand{\kb}{\ensuremath{k\sub{B}}}
\newcommand{\NA}{\ensuremath{\text{NA}}}
\newcommand{\BW}{\ensuremath{\text{BW}}}
\newcommand{\SNR}{\ensuremath{\text{SNR}}}
\newcommand{\pathloss}{\ensuremath{\text{PL}}}

\usepackage[urw-garamond]{mathdesign}
\delimitershortfall=1pt
\delimiterfactor=1100
\usepackage[kerning=true]{microtype}
\SetExtraKerning{encoding={T1}}{\textemdash={215,215}}
%\SetProtrusion[load=ugm-T1,context=biblio]{encoding=T1,family=ugm}{\textquotedblleft={80,80}}
\usepackage{textcomp}
\newcommand\od{\text{d}}

\usepackage{tikz}
%\usepackage{pgfplots}

%\usetikzlibrary{arrows,shapes,shapes.symbols,calc,positioning,matrix,chains}
\usetikzlibrary{arrows,arrows.new,shapes.geometric,shapes.symbols,matrix,calc,decorations.pathmorphing,decorations.pathreplacing,decorations.markings,decorations.shapes,fit}
\pgfdeclarelayer{background}
\pgfdeclarelayer{foreground}
\pgfsetlayers{background,main,foreground}

\usepackage{adjustbox}
%\usepackage{eqparbox}
%\newsavebox{\tstretchbox}
%\newcolumntype{S}[1]{>{\begin{lrbox}{\tstretchbox}}l<{\end{lrbox}\eqmakebox[#1][s]{\unhcopy\tstretchbox}}}

\usepackage[ps2pdf,breaklinks=true,bookmarks=true,bookmarksopen,bookmarksopenlevel=2,pdfpagelayout=OneColumn]{hyperref}
\usepackage[top=0.88in,headsep=5mm,bottom=0.6in,right=0.5in,left=0.5in]{geometry}
\usepackage{cleveref}
\usepackage{url}
\usepackage{breakurl}

\usepackage{csquotes}
\MakeAutoQuote{«}{»}
%\usepackage[numbers,sort&compress]{natbib}
\usepackage[
    backend=biber,
    style=numeric-comp,
    sorting=ydnt,
    sortcites=true,
    natbib=true,
    url=true, 
    doi=true,
    eprint=true,
    backref=true,
    backrefstyle=three,
    maxnames=6,
    minnames=5,
    defernumbers=true,
]{biblatex}
\usepackage{xpatch}
\setlength{\bibitemsep}{0.7ex plus 0.25ex minus 0.2ex}
\renewcommand{\subtitlepunct}{\addcolon\space}
\renewcommand{\intitlepunct}{\space}
\renewcommand{\newunitpunct}{\addcomma\space}
\renewcommand{\newblockpunct}{\unspace\penalty-8\relax\space}
\renewcommand{\labelnamepunct}{\addperiod}
%\renewcommand{\multinamedelim}{\addsemicolon\space}
%\DefineBibliographyExtras{american}{\renewcommand{\finalandcomma}{\addsemicolon}}
%\renewcommand{\relatedpunct}{\space}
%\renewcommand{\relateddelim}{\printtext{\space and\space\nopunct}}
\renewcommand{\mkbibnamelast}[1]{\textsc{#1}}
\DeclareNameAlias{default}{last-first/first-last}
\renewbibmacro{name:last-first}[4]{%
  \ifuseprefix
    {\usebibmacro{name:delim}{#3#1}%
     \usebibmacro{name:hook}{#3#1}%
     \ifblank{#3}{}{%
       \ifcapital
         {\mkbibnameprefix{\MakeCapital{#3}}\isdot}
     {\mkbibnameprefix{#3}\isdot}%
       \ifpunctmark{'}{}{\bibnamedelimc}}%
     \mkbibnamelast{{\bfseries #1}}\isdot
     \ifblank{#4}{}{\bibnamedelimd\mkbibnameaffix{#4}\isdot}%
     %\ifblank{#2}{}{\revsdnamepunct\bibnamedelimd\mkbibnamefirst{#2}\isdot}
	}
    {\usebibmacro{name:delim}{#1}%
     \usebibmacro{name:hook}{#1}%
     \mkbibnamelast{{\bfseries #1}}\isdot
     \ifblank{#4}{}{\bibnamedelimd\mkbibnameaffix{#4}\isdot}%
     %\ifblank{#2#3}{}{\revsdnamepunct}%
     \ifblank{#2}{}{\bibnamedelimd\mkbibnamefirst{#2}\isdot}%
     \ifblank{#3}{}{\bibnamedelimd\mkbibnameprefix{#3}\isdot}}}
\DeclareFieldFormat{pages}{\mkcomprange{#1}}
\DeclareFieldFormat{sentencecase}{\MakeSentenceCase{#1}}
\DeclareFieldFormat{sortyear}{\textbf{#1}}
\renewbibmacro*{title}{%
  \ifthenelse{\iffieldundef{title}\AND\iffieldundef{subtitle}}
    {}
    {\ifthenelse{\ifentrytype{article}\OR\ifentrytype{inbook}%
      \OR\ifentrytype{incollection}\OR\ifentrytype{inproceedings}%
      \OR\ifentrytype{inreference}}
      {\printtext[title]{%
        \printfield[sentencecase]{title}%
        \setunit{\subtitlepunct}%
        \printfield[sentencecase]{subtitle}}}%
      {\printtext[title]{%
        \printfield[titlecase]{title}%
        \setunit{\subtitlepunct}%
        \printfield[titlecase]{subtitle}}}%
     \newunit}%
  \printfield{titleaddon}}
\xapptobibmacro{pageref}{\renewcommand{\newunitpunct}{\addsemicolon\space}}{}{}
\xapptocmd{\bibsetup}{\flushbottom\raggedright}{}{}
%\xapptocmd{\bibfont}{\small}
%\newbibmacro*{related:prelim}[1]{\entrydata{#1}{\mkbibparens{\usebibmacro{date}}\nopunct\space\bibstring{on}\usebibmacro{doi+eprint+url}}}
\newbibmacro*{related:prelim}[1]{\entrydata{#1}{\usebibmacro{doi+eprint+url}\setunit{\addspace}\printtext{\bibopenparen}\usebibmacro{date}\iffieldundef{note}{}{\newunit\printfield{note}}\printtext{\bibcloseparen}}}
\NewBibliographyString{prelim,prelims,on}
\DefineBibliographyStrings{american}{%
  in = {{}},
  url = {{URL}},
  backrefpage = {{cited on p. }},
  backrefpages = {{cited on pp. }},
  prelim = {preliminary version on\nopunct},
  prelims = {preliminary versions at\nopunct},
  on = {on\nopunct},
}

\newcommand\anref[1]{\hyperref[#1]{\autoref*{#1} (\nameref*{#1})}}
%\usepackage{doi}
%\usepackage{makeidx}

\hypersetup{%
    unicode=false,          % non-Latin characters in Acrobat’s bookmarks
    pdftoolbar=false,        % show Acrobat's toolbar?
    pdfmenubar=true,        % show Acrobat's menu?
    pdffitwindow=true,     % window fit to page when opened
    pdfstartview={FitBV},    % fits the width of the page to the window
    pdfauthor={David Dalrymple},     % author
    pdfcreator={David Dalrymple},   % creator of the document
    pdfproducer={David Dalrymple}, % producer of the document
    pdfnewwindow=true,      % links in new window
    colorlinks=true,       % false: boxed links; true: colored links
    linkcolor=blue,          % color of internal links
    citecolor=blue,        % color of links to bibliography
    filecolor=magenta,      % color of file links
    urlcolor=cyan           % color of external links
}
\newcommand\figref[1]{\hyperref[#1]{Figure~\ref{#1}}}

% fancy quotes
\definecolor{quotemark}{gray}{0.7}
\makeatletter
\def\fquote{%
    \@ifnextchar[{\fquote@i}{\fquote@i[]}%]
           }%
\def\fquote@i[#1]{%
    \def\tempa{#1}%
    \@ifnextchar[{\fquote@ii}{\fquote@ii[]}%]
                 }%
\def\fquote@ii[#1]{%
    \def\tempb{#1}%
    \@ifnextchar[{\fquote@iii}{\fquote@iii[]}%]
                      }%
\def\fquote@iii[#1]{%
    \def\tempc{#1}%
    \vspace{-0.15em}%
    \noindent%
    \begin{list}{}{%
 	   %
         \setlength{\leftmargin}{0.12\textwidth}%
         \setlength{\rightmargin}{0.12\textwidth}%
                  }%
         \item[]%
         \begin{picture}(0,0)%
         \put(-21,-5){\makebox(0,0){\scalebox{4.4}{\selectfont\textcolor{quotemark}{``}}}}%
         \end{picture}%
         \begingroup}%
         
 \def\endfquote{%
 \endgroup\par%
 \makebox[0pt][l]{%
 \hspace{0.8\textwidth}%
 \begin{picture}(0,0)(0,0)%
 \put(0.6,14){\makebox(0,0){%
 \scalebox{4.4}{\selectfont\color{quotemark}''}}}%
 \end{picture}}%
 \vskip -2.7em%
 \ifx\tempa\empty%
 \else%
    \ifx\tempc\empty%
       \hfill\rule{100pt}{0.25pt}\\\mbox{}\hfill{\small\tempa,\ \emph{\tempb}}%
   \else%
       \hfill\rule{100pt}{0.25pt}\\\mbox{}\hfill{\small\tempa,\ \emph{\tempb},\ \tempc}%
   \fi\fi\par%
   \vspace{0.63em}%
 \end{list}%
 }%
 \makeatother

%\DeclareMathOperator*{\argmax}{arg\:max}
%\DeclareMathOperator*{\argmin}{arg\:min}

\usepackage{subscript}

\newcommand{\attrib}[1]{\nopagebreak{\raggedleft\footnotesize #1\par}}
\newcommand{\todo}[1]{\textcolor{lightgray}{\textit{<<#1>>}}}
\newcommand{\tbc}{\begin{center} \todo{to be completed} \end{center}}
  \newcommand{\notes}[3]{\vspace{1.5cm}\begin{center} \parbox[b]{3cm}{\hfill #1 \hfill} \hspace{2cm} \textbf{#3} \hspace{2cm} \parbox[b]{3cm}{\hfill \textit{#2} \hfill} \\[-3mm] \rule{\textwidth}{0.4pt} \end{center}}

\flushbottom
\setlength{\parskip}{0.3cm plus1mm minus1mm}
\setlength{\parindent}{0cm}
\setlength{\columnsep}{7mm}
%\setlength{\premulticols}{1cm}
%\setlength{\postmulticols}{1cm}

\pgfmathsetseed{519}
\newcommand{\myprod}{\color{purple!60!blue}\hskip0.08pt$\bigotimes$}
\colorlet{erred}{red!75!black!75!white}
\colorlet{modelc}{cyan!64!blue}
%\addbibresource{neuro_theory.bib}

\title{Information-Theoretic, Probabilistic, and Algorithmic Considerations for the Design of Brain Activity Imaging Experiments}

%\author[1,2]{\ \lift{$\jointfirst\,$}Adam~H.~Marblestone\rlap{,}}
\author[1,2]{David~A.~Dalrymple}

%\affil[$\jointfirst$]{Joint first authors}
%\affil[$\jointlast$]{Joint last authors}

\newcommand\et{{\em \&}}

\affil[1]{Nemaload, San Francisco, CA{~94107, USA}}
\affil[2]{Media Lab{oratory,} Massachusetts Institute of Technology{, Cambridge,~MA~02139, USA}}

\renewcommand{\maketitlehookc}{{\small\raggedright Correspondence to: \texttt{david\,\textnormal{(at)}\,\,dalrymple.co}}}

\begin{document}
\maketitle
\pagestyle{plain}
\thispagestyle{empty}

\section{Introduction}

\textsc{Imaging} can be defined as the estimation, from sensor readings, of a physical quantity that varies over some particular region of space (and possibly time). In many cases (including almost all medical imaging, except for the humble X-ray), this is a nontrivial inference problem.

In the case of ``brain activity mapping,'' imaging is only one piece of the overall puzzle, which is illustrated in \autoref{fig:puzzle}.

%The neural computation should be estimated as a stochastic differential equation, of the type $\mbox{Phase Space} \rightarrow \left(\mbox{$\sigma$-Algebra of Tangent Space} \rightarrow \mathbb{R}\right)$.

%Neural Computation: $X_{\mbox{ph}} \rightarrow \left( \sigma(X_{\mbox{ph}}^*) \rightarrow \mathbb{R} \right)$ \\

\begin{landscape}
\begin{figure}[p]
\caption{An overview of variables involved in an electro-optical measurement of neural computation.}
\label{fig:puzzle}
\begin{center}
\adjustbox{lap=0cm,scale=1.75}{
    \begin{tikzpicture}[
    line width=0.35pt,
    every node/.style={draw=black,align=center, cloud ignores aspect,line width=0.2pt,rounded corners=0.8pt,scale=0.34},
    every matrix/.style={row sep=3ex,column sep=2pt},
    every fit/.style={sharp corners,draw=none},
    prod/.style={scale=1.22,inner ysep=0.4pt,inner xsep=0.4pt,line width=0.001pt,draw=none},
    fof/.style={draw=blue!60!green,fill=white},
    choice/.style={fill=white},
    math/.style={draw=yellow,line width=0.42},
    unk/.style={color=red!75!black,draw=red!50!black!20!white,fill=white},
    sq/.style={draw=green!60!black!20!white,fill=white},
    blank/.style={draw=none},
    understudy/.style={color=blue!80!white!80!black,font=\bfseries,line width=0.32pt,fill=white},
    sh/.style={color=blue!60!green,font=\bfseries},
    shd/.style={color=blue!60!green!#1!erred},
    varia/.style={line width=0.24,color=red!60!black!95!white},
    proj/.style={line width=0.43,blue!42!green!42!white, dash pattern=on 2pt off 2pt,postaction={draw,blue!66!green!50!white,dash pattern=on 2pt off 2pt,dash phase=2pt},-},
    proj2/.style={line width=0.43,black!25!white, dash pattern=on 2pt off 2pt,postaction={draw,black!40!white,dash pattern=on 2pt off 2pt,dash phase=2pt},-},
    rbdash/.style={line width=0.5,draw=red!85!black, dash pattern=on 3pt off 3pt,postaction={draw,black,dash pattern=on 3pt off 3pt,dash phase=3pt}},
    control/.style={double=white,line width=0.3pt},
    trycontrol/.style={decorate,decoration={saw,amplitude=0.8pt,segment length=3pt,post length=5pt,pre length=1pt}},
    errarr/.style={color=erred},
    opterr/.style={double=red!95!black!75!white,double distance between line centers=0.6pt,line width=0.3pt},
    photon/.style={decorate,decoration={snake,amplitude=0.8pt,segment length=2.8pt,post length=4pt,pre length=1pt},line width=0.33pt},
    readn/.style={decorate,decoration={random steps,amplitude=0.8pt,segment length=1.3pt,post length=4pt,pre length=1pt}},
    thermo/.style={decorate,decoration={random steps,amplitude=0.16pt,segment length=1pt,post length=2pt},line width=0.24pt},
    mechn/.style={decorate,decoration={zigzag,amplitude=0.8pt,segment length=5pt,post length=4pt,pre length=1pt},line width=0.26},
    squirm/.style={decorate,decoration={random steps,amplitude=1.5pt,segment length=5pt,post length=4pt,pre length=1pt}},
    roundoff/.style={decorate,decoration={bent},line width=0.29},
    episteme/.style={double=red!80!black!70!white,double distance between line centers=1.7pt,postaction={color=yellow,decorate},decoration={markings,mark=between positions 0.1 and 0.9 step 1ex with {\node[draw=none] {?};}},arrow head=4.9pt,>=stealth new},
    thought/.style={draw=black!60!yellow,double=yellow,>=stealth new,arrow head=4pt},
    understudy_arr/.style={color=blue!80!white!80!black},
    sigchain/.style={color=blue!80!white!80!black,>=latex'},
    sidechain/.style={color=blue!80!white!80!black,>=latex',line width=0.3},
    behavior/.style={-,line width=1.0pt,color={green!60!black!13!white},postaction={decorate,decoration={shape backgrounds,pre length=1.5pt, post length=1.5pt, shape=isosceles triangle, shape start height=1.0pt, shape start width=2.4pt, shape sep={4.54pt, between centers}}, line width=0.0pt,draw=none,fill=green!60!black!45!white}},
    feedback/.style={-,line width=1.0pt,color={green!20!blue!60!black!13!white},postaction={decorate,decoration={shape backgrounds,pre length=1.5pt, post length=1.5pt, shape=isosceles triangle, shape start height=1.0pt, shape start width=2.4pt, shape sep={5.1pt, between centers}}, line width=0.0pt,draw=none,fill=green!20!blue!60!black!45!white}},
    mfeedback/.style={-,line width=1.4pt,color={red!40!blue!70!black!13!white},postaction={decorate,decoration={shape backgrounds,pre length=1.5pt, post length=1.5pt, shape=isosceles triangle, shape start height=1.4pt, shape start width=2.4pt, shape sep={5.1pt, between centers}}, line width=0.0pt,draw=none,fill=red!40!blue!70!black!45!white}},
    ]
            \begin{pgfonlayer}{foreground}
            \matrix[draw=none,ampersand replacement=\t] (m) at (6.72,0) {
                \t \t \t[4.4pt] \coordinate (nex3); \\[-8pt]
                \t \t \t \t[15pt] \t[-6pt] \t[6pt] \t \t \node[fof] (light) {Light Source}; \t \t[12pt] \t \t[9pt]\\[-9pt]
                \t \node[prod] (geneticvar) {\myprod}; \t \t \t \t \t \t \t \t \t \t \coordinate (nex2); \t \node[choice,math] (learner) {Active \\ Learning}; \t \node[prod] (expdes) {\myprod};  \\[-10pt]
                \t \t \t \t \t \t \coordinate (nex); \t \t \t \t \node[choice,math] (analysis) {Analysis}; \t \t  \t  \\[-11.7pt]
                \t \node[choice] (genetics) {Genetic \\ Engineering}; \t \node[fof] (unwanted) {Unwanted \\ Perturbations \\ \textit{(e.g. Toxicity)}};\t \node[choice,opacity=0.5,draw=gray,rbdash,line width=0.27] (perturb) {Controlled \\ Perturbations}; \t \t  \node[fof,fill=white,rbdash,line width=0.24] (indicator) {Physical \\ Indicator};  \t \node[choice] (immob) {Immobilization}; \t \node[choice] (tracking) {Tracking \\ Stage}; \t \node[choice,draw=blue!60!green!85!white] (optics) {Optical \\ System};\t \node[fof] (sensor) {Sensor};\t\t \node[fof] (estamove) {Estimated \\ Specimen \\ Movement};  \t \node[choice,math] (sysid) {System \\ Identification}; \\[-2.8pt]
                \t \node[understudy] (neuralcomp) {Abstract \\  Neural \\ Computation};
                \t \node[prod] (actualcomp) {\myprod};
                \t \node[prod] (statehisp) {\myprod};
                \t \node[sh,fof,label={[align=flush center,text width=77pt]below:(Presumed one time series per each neuron.)}] (statehist) {Observable State \\ History};
                \t \node[prod] (fluor) {\myprod};
                \t \node[prod] (fluorposed) {\myprod};
                \t \node[prod] (fluorworld) {\myprod};
                \t \node[prod] (insplane) {\myprod};
                \t \node[prod] (frames) {\myprod};
                \t \node[prod] (estap) {\myprod};
                \t \node[fof,font=\bfseries,color=blue!65!green,fill=white] (esta) {Estimated \\ State \\ History};
                \t \node[prod] (estcomp) {\myprod};
                \t \node[fof,font=\bfseries,color=modelc,fill=white] (model) {Estimated \\ Neural Computation \\ (Model)};\\[-14pt]
                \t \node[unk] (varia) {Biological \\ Variability};
                \t\node[unk] (stochast) {Neural \\ Stochasticity};
                \t \node[unk] (squish) {Uncontrolled \\ Environmental \\ Inputs};
                \t
                %\t \node[unk] (thermo) {Thermodynamic \\ Noise};
                \t \node[unk] (autofluor) {Autofluorescence \\ / Background};
                \t \node[unk,sq] (squirm) {Specimen \\ Movement};
                \t \node[unk] (mechn) {Mechanical \\ Noise};
                \t \node[unk] (shot) {Shot \\ Noise};
                \t \node[unk] (readn) {Read \\ Noise};
                \t \node[unk] (roundoff) {Roundoff \& \\ Approximations};
                \t
                \t \node[unk,draw=yellow!70!gray] (epistemic) {Epistemic \\ Uncertainty};\\
            };
            \end{pgfonlayer}

            \coordinate (arrowdrop) at ($ (varia.west)+(-3.9mm,0) $);

            \begin{scope}[line width=0.4pt,->,>=stealth new]
            \draw [understudy_arr] (neuralcomp) -- (actualcomp);
            \draw[varia] (unwanted) -- (actualcomp);
            \draw (stochast) -- (actualcomp);
            \draw[sigchain] (actualcomp) -- node[below,pos=0.416,draw=none,scale=0.9] {``Actual'' Neural\\Computation} (statehisp);
            \draw[trycontrol,color=orange!40!white] (perturb) -- (statehisp);
            \draw[color=black!40!white] (perturb) -- (unwanted);
            \draw[varia] (squish) -- (statehisp);
            \draw[sigchain,sh,-,line width=0.66pt] (statehisp) -- (statehist);

            \draw[sigchain,sh,line width=0.66pt] (statehist) -- (fluor);
            \draw[errarr,arrow head=2.5pt] (autofluor) -- (fluor);
            \draw[errarr,arrow head=2.5pt] (autofluor.east) -- (fluorposed);
            \draw[trycontrol,draw=black!64!white,opacity=0.6] ($ (immob.south) + (-5pt,0) $) .. controls ($ (fluorposed.west) + (-17pt,6pt) $) and ($ (squirm.north) + (-1pt,15pt) $)  .. ($ (squirm.north) + (-1pt,0) $);
            \draw[sigchain,shd=90,line width=0.63pt] (fluor) -- node[above=1.7pt,draw=none,text width=90pt,align=flush center,pos=0.44,scale=0.97] {Signal in\ldots} node[below,draw=none,text width=60pt,align=flush center,pos=0.42] {\small $f(u,v,w)$:  \textbf{Brain} Coordinates} (fluorposed);
            \draw[squirm,errarr] ($ (squirm.north) + (2pt,0) $) -- (fluorposed);
            \draw[sigchain,shd=70,line width=0.53pt] (fluorposed) -- node[below,draw=none,text width=60pt,align=flush center] { {\tiny $f(x-x_0,y-y_0,z-z_0)$:} \textbf{Relative} Coordinates} (fluorworld);
            \draw[control] (tracking) -- (fluorworld);
            \draw[mechn,errarr] (mechn) -- (fluorworld);

            \draw[behavior] ($ (statehisp) + (15.5pt,-0.2pt) $) |- node[pos=0.75,above,draw=none] (behl) {\color{green!60!black!58!white}\textsc{Motor Command}} ($ (squirm.south) + (-3pt,-15pt) $) -- ($ (squirm.south) + (-3pt,0) $);
            \draw[behavior] ($ (squirm.south) + (2.3pt,0) $) -- ++(0pt,-20.3pt) node[coordinate] (nex5) {} -| node[pos=0.25,below,draw=none] (prol) {\color{green!60!black!48!white}\textsc{Proprioception}} ($ (squish.south)!(statehisp)!(squish.south east) + (10pt,0) $);
            \coordinate (nex4) at ($ (squish.south) + (15.5pt,0) $);
            \coordinate (nex6) at ($ (squish.south) + (10pt,-12pt) $);
            \begin{scope}[color=green!60!black!54!white,->,line width=0.25pt,arrow head=2.2pt]
                \draw ($ (behl.mid west) + (0,0.38pt) $) -- ++(-6.5pt,0) ($ (behl.mid east) + (0,0.38pt) $) -- ++(7pt,0);
                \draw ($ (prol.mid east) + (0,0.38pt) $) -- ++(7.5pt,0) ($ (prol.mid west) + (0,0.38pt) $) -- ++(-8pt,0);
            \end{scope}

            \draw [sigchain,shd=60,line width=0.49pt] (fluorworld) -- node[below,draw=none,text width=60pt,align=center,pos=0.45,scale=0.95] { $f(x,y,z)$: \textbf{Absolute} Coordinates} (insplane);
            \draw [opterr] (optics) -- (insplane);
            \draw [proj] (optics) -- (sensor);
            \draw [photon,errarr] (shot) -- (frames);
            \draw [sigchain,shd=40,line width=0.33pt] (insplane) -- node[above=2pt,draw=none,text width=50pt,align=center,pos=0.46] {$f(x_s,y_s)$: \textbf{Sensor} Coordinates} (frames);
            \draw [opterr] (sensor) -- (frames);
            \draw [readn,errarr] (readn) -- (frames);

            \draw [sigchain,shd=30, line width=0.24pt] (frames) -- node[below,draw=none,text width=40pt,align=center,pos=0.215] {$f(i,j)$: \textbf{Pixels}} (estap);
            \draw [thought] (analysis) -- (estap);
            \draw [roundoff,errarr] (roundoff) -- (estap);
            \draw [sigchain,shd=90] (estap) -- (esta);
            \draw [sidechain,shd=95] ($ (estap.east)!0.45!(esta.west) $) |- (estamove);
            \draw [sigchain,shd=90] (esta) -- (estcomp);
            \draw[episteme] (epistemic) -- (estcomp);
            \draw [thought] (sysid) -- (estcomp);
            \draw [sigchain] (estcomp) -- (model);

            \draw [thought] (learner) -- (expdes);
            \draw [color=modelc] (model) -- (expdes);
            \draw[mfeedback] (expdes) |- node[pos=0.75,above,draw=none] (expctrl) {\color{red!40!blue!70!black!75!white}\textsc{Experimental Design}}  ($ (nex3) + (-3pt,0) $) -- ($ (perturb.north) + (-4pt,0) $);
            \begin{scope}[color=red!40!blue!70!black!75!white,->,line width=0.25pt,arrow head=2.2pt]
                \draw ($ (expctrl.mid east) + (0,0.38pt) $) -- ++(7.2pt,0) ($ (expctrl.mid west) + (0,0.38pt) $) -- ++(-8pt,0);
            \end{scope}

            \draw[feedback] (estamove) -- (nex2) -| node[pos=0.284,above,draw=none] (ctrl) {\color{green!20!blue!60!black!45!white}\textsc{Control}} ($ (tracking.north) + (-3pt,0) $);
            \draw[feedback] ($ (tracking.north) + (3pt,0) $) |- node[pos=0.828,below,draw=none] (fbak) {\color{green!20!blue!60!black!45!white}\textsc{Feedback}} ($ (estap.west) + (-9pt,40.7pt) $) -- ++(0,-40.5pt);
            \begin{scope}[color=green!20!blue!60!black!45!white,->,line width=0.25pt,arrow head=2.2pt]
                \draw ($ (fbak.mid west) + (0,0.38pt) $) -- ++(-7pt,0) ($ (fbak.mid east) + (0,0.38pt) $) -- ++(7pt,0);
                \draw ($ (ctrl.mid east) + (0,0.38pt) $) -- ++(7.2pt,0) ($ (ctrl.mid west) + (0,0.38pt) $) -- ++(-7.4pt,0);
            \end{scope}

            \draw (genetics) -- (geneticvar);
            \draw[densely dotted] (varia) -- (neuralcomp);
            \draw[varia] (varia) -| (arrowdrop) |- (geneticvar);
            \draw[rbdash,line width=0.28] (light) -| (geneticvar);
            \draw[proj2] (immob.north) -- (nex) -| ($ (perturb.north) + (4pt,0) $);
            \draw [rbdash] ($ (geneticvar -| unwanted) $) -- (unwanted.north);
            \draw [rbdash,opacity=0.4] ($ (geneticvar -| perturb) $) -- (perturb.north);
            \draw [rbdash] (geneticvar) -| (fluor);
            \draw [proj] (optics) -- (light);
            \draw [color=yellow,-,line width=0.26] (learner) -- (sysid) |- (analysis);
        
            \coordinate (anacorner) at (ctrl -| analysis);
            \coordinate (anacorner2) at ($ (tracking.north) + (-3pt,18pt) $);
            \coordinate (leacorner) at ($(perturb.south west)!(nex3)!(perturb.north west) $);
            \end{scope}


        \begin{pgfonlayer}{background}
        \begin{scope}[every node/.style={fill=black!6!white!92!blue!60!white}]
            \node[fit=(neuralcomp.north west) (squish.south east) (nex4)] (nnn1) {};
            \node[fit=(behl.center) (prol.center) (nex5) (nex6)] (nnn2) {};
            \node[fit=(nnn2.south west) (nnn1.north east),inner sep=0pt] {};
            %\node[fit=(nex6) (nex5) ] {};
        \end{scope}
        \begin{scope}[every node/.style={fill=black!14!white!92!orange}]
            \node[fit=($ (statehist.west)!0.05!(statehisp.east) $) (indicator.north west) (autofluor.south west) (sensor.north east) (readn.south east) ($ (frames.east)!0.45!(estap.west) $)] {};
            %\node[fit=(light.north west) (light.north east) (optics.south)] {};
        \end{scope}
        \begin{scope}[every node/.style={fill=black!6!white}]
            \node[fit=(nex2) (roundoff.south west) (estamove.east) (anacorner)] {};
            \node[fit=(anacorner) (fbak.base west) (anacorner2)] {};
        \end{scope}
        \begin{scope}[every node/.style={fill=black!13!white}]
            \node[fit=(model.east) (sysid.north west) (epistemic.south west) ] {};
        \end{scope}
        \begin{scope}[every node/.style={fill=black!22!white}]
            \node[fit=($ (sysid.south west)!(learner.south west)!(sysid.north west) $) (expdes.east) ($ (expdes.south)!(expctrl)!(expdes.north) $)] {};
            \node[fit=($ (expdes.south east)!(expctrl)!(expdes.north east) $) (leacorner) ] {};
            \node[fit=(leacorner) (perturb.south west) (perturb.south east)] {};
        \end{scope}
        \end{pgfonlayer}

    \end{tikzpicture}
}
\end{center}
\end{figure}
\end{landscape}


Let us focus on the imaging problem for now.

\subsection{Forward Model}

\subsubsection{Emission: from Discrete Signals to Fluorescence in Cartesian Space}

Here we begin with an
``{\color{blue!60!green}Observable State History},'' presumed to consist of one
time-varying signal for each neuron. We define $\mathcal{S} \in
\left[C(\Re_{\ge 0})\right]^n$ such that $\mathcal{S}_i(t)$ are these
time-varying signals. Next, the Physical Indicator associates each point in
$(0,1)^3$ (the specimen coordinate system) with a single such time series, multiplied by a factor (proportional to the local density of emitted photon flux) and convolved
with characteristic time behavior: $\mathcal{P} \in
C\!\left[(0,1)^3,\bigvee_{i=0}^{n-1} \Re_{\ge 0}\right] \times
C^\infty_C(\Re_{\ge 0})$, such that $f_0(u,v,w) = \left(\mathcal{S} \cdot \mathcal{P}_0(u,v,w) \right) \ast \mathcal{P}_1$, i.e. $f_0 = \left(\mathcal{P}_1 \,\ast\right) \circ \left(\mathcal{S} \, \cdot\right) \circ \mathcal{P}_0$. We then add in time-invariant background, $\mathcal{B}_0 \in C\left[(0,1)^3,\Re_{\ge 0}\right]$, such that $f_b(u,v,w)(t)=f_0(u,v,w)(t)+\mathcal{B}_0(u,v,w)$, or simply $f_b = f_0 + \mathcal{B}_0$.
Next, the Specimen Movement induces a time-dependent embedding of the unit $(u,v,w)$ cube (considered to parameterize the anterior-posterior, dorsal-ventral, and left-right axes of the specimen) into ``relative'' coordinates (i.e. a Cartesian coordinate system referenced to the specimen-holder). We define $\mathcal{Q} \in C\left[\Re_{\geq 0}, C^1\left[(0,1)^3,\Re^3\right]\right]$. Now, for any bounded
subset $S \subset (0,1)^3$, we want $\int_S f_b = \int_{\mathcal{Q}(t)(S)} f_1$. By the change of variables theorem,
\[\int_{\mathcal{Q}(t)(S)} f_1 = \int_S \left(f_1 \circ \mathcal{Q}(t)\right) \left|\det D\mathcal{Q}(t)\right|\]
\[\frac{1}{\left|\det D\mathcal{Q}(t)\right|} f_b = f_1 \circ \mathcal{Q}(t) \]
\[f_1(x,y,z) = \begin{dcases} \frac{1}{\left|\det D\mathcal{Q}(t)\right|} f_b \circ \mathcal{Q}(t)^{-1} & \mbox{if } (x,y,z) \in \mathcal{Q}(t)\left((0,1)^3\right) \\ 0 & \mbox{otherwise} \end{dcases}\]
Finally, we add in another time-invariant background, this one fixed in $(x,y,z)$ space, $\mathcal{B}_1 \in C\left[\Re^3,\Re_{\geq 0}\right]$ with $f_2 = f_1 + \mathcal{B}_1$. The expanded equation for fluorescence in specimen-holder coordinates is:
\[f_2(x,y,z)(t) = \mathcal{B}_1(x,y,z) + \begin{dcases} \dfrac{\mathcal{B}_0(u,v,w) + \int_{-\infty}^{\infty} d\tau \mathcal{P}_1(t-\tau) \left( \mathcal{S}(\tau) \cdot \mathcal{P}_0(u,v,w)\right) }{\left|\det D\mathcal{Q}(t)(u,v,w)\right|} & \mbox{if } (x,y,z) = \mathcal{Q}(t)(u,v,w) \mbox{, } (u,v,w)\in(0,1)^3 \\ 0 & \mbox{otherwise}\end{dcases}\]

\subsubsection{Imaging: from Fluorescence to Sensor Intensity}

The position and orientation of the specimen-holder with respect to the optical system are characterized by $\mathcal{T} \in C\!\left[\Re_{\ge 0}, \Re^3 \times SO(3)\right]$, such that $f_r\left(\mathcal{T}_1(t) \vec{x} - \mathcal{T}_0(t)\right)(t) = f_2(\vec{x})(t) $. We may write $\mathcal{T}_1(t) \vec{x} - \mathcal{T}_0(t)$ as $\mathcal{T}(t)(\vec{x})$ or simply $\vec{r}$, and the inverse $\mathcal{T}_1(t)^{-1} ( \vec{r} + \mathcal{T}_0(t) )$ as $\mathcal{T}^{-1}(t)(\vec{r})$, so that $f_r(\vec{r})(t) = f_2(\mathcal{T}^{-1}(t)(\vec{r}))(t)$.

The input and output intensities of any passive optical system are related by a Fredholm integral equation:
\[f_s(\vec{s})(t) = \int d\vec{r}\,K(\vec{r},\vec{s})\,f_r(\vec{r})(t)\]
We also consider active optical systems, which we model using a time-dependent Fredholm kernel:
\[f_s(\vec{s})(t) = \int d\vec{r}\,K(\vec{r},\vec{s},t)\,f_r(\vec{r})(t)\]

The expanded equation for sensor intensity in a generic active optical system is:
\[f_s(\vec{s})(t) = \int_{\Re^3} K(\mathcal{T}^{-1}(t)(\vec{x}),\vec{s},t)\,\mathcal{B}_1\!\left(\vec{x}\right) d\vec{x} + \int_{(0,1)^3} K\left(\left(\mathcal{T}\circ\mathcal{Q}\right)(t)(\vec{u}),\vec{s},t\right) \frac{\mathcal{B}_0(\vec{u}) + \int_{-\infty}^{\infty} d\tau \mathcal{P}_1(t-\tau) \left( \mathcal{S}(\tau) \cdot \mathcal{P}_0(\vec{u})\right) }{\left|\det D\mathcal{Q}(t)(\vec{u})\right|} d\vec{u} \]

\subsubsection{Measurement: from Sensor Intensity to Data}

The sensor comprises a finite grid of \textit{pixels}, each of which induces shot noise, read noise, and discretization
(in space, time, and value) on the local sensor intensity. Specifically,
\[D_{ijk} = \max 0, \min 2^{16}\!-1, \left\lfloor\mathcal{L}_0+\mathcal{L}_1 \cdot \left(\mathcal{R}_{ijk} + \mathcal{C}_{ij} \cdot \left(t_{k,1}-t_{k,0}\right) + \mathcal{E}\cdot\mathrm{Poisson}^{-1}\left[\mathcal{U}_{ijk},\int_{t_{k,0}}^{t_{k,1}} \int_{\alpha_{ij}} f_s(\vec{s})(t)\, d\vec{s}\,dt\right]\right)\right\rfloor\]
where $\alpha_{ij}$ is the area of $\vec{s}$-space occupied by pixel $i,j$, $\mathcal{U}_{ijk}\in(0,1)$ represents the shot noise, $\mathcal{E} \in (0,1)$ represents the quantum efficiency (average electrons per photon), $\mathcal{C}_{ij}\in\Re_{\ge 0}$ is the dark current of pixel $i,j$ (electrons per second), $\mathcal{R}_{ijk} \in \Re$ represents the read noise, $\mathcal{L} \in \Re_{\ge 0}^2$ is a linear model of the amplifier and A/D, and $D_{ijk} \in 2^{16}$ is a datum (the value of pixel $i,j$ in frame $k$). Further, we assume the priors $\mathcal{R}_{ijk} \sim \mathrm{Gaussian}(0, \mathcal{R}_{ij})$, $\mathcal{R}_{ij} \sim \mbox{L\'evy}(\mathcal{R}_\mu,\mathcal{R}_c)$ ($\mathcal{R}_\mu\approx 0.8$, $\mathcal{R}_c\approx 0.4$ estimated from Andor Neo datasheet), and $\mathcal{U}_{ijk} \sim \mathrm{Uniform}(0,1)$.

\subsection{Induced Constraints}

Given a full complement of $\{D_{ijk}\}$ (where $i \in W$, $j \in H$, and $k \in T$), what conclusions can we draw? Let us begin by introducing new variables $I_{ijk} \in \Re_{\ge 0}$ (the intensity at pixel $(i,j)$ integrated over frame $k$) and $G_{ijk}\in\mathbb{N}$ (the number of photons actually detected within pixel $i,j$ during the interval of frame $k$) to eliminate the special function $\mathrm{Poisson}^{-1}$:
\[D_{ijk} = \max 0, \min 2^{16}\!-1, \left\lfloor\mathcal{L}_0+\mathcal{L}_1 \cdot \left(\mathcal{R}_{ijk} + \mathcal{C}_{ij} \cdot \left(t_{k,1}-t_{k,0}\right) + \mathcal{E}\cdot G_{ijk}\right)\right\rfloor\]
%\[\frac{1}{G_{ijk}!} \int_{I_{ijk}}^{\infty} t^{G_{ijk}} e^{-t} dt \leq \mathcal{U}_{ijk} < \frac{1}{(G_{ijk}+1)!} \int_{I_{ijk}}^{\infty} t^{G_{ijk} + 1} e^{-t} dt \]
\[e^{-I_{ijk}} \sum_{z=0}^{G_{ijk} - 1} \frac{\left(I_{ijk}\right)^z}{z!} \leq \mathcal{U}_{ijk} < e^{-I_{ijk}} \sum_{z=0}^{G_{ijk}} \frac{\left(I_{ijk}\right)^z}{z!}\]
\[\int_{t_{k,0}}^{t_{k,1}} \int_{\alpha_{ij}} f_s(\vec{s})(t)\, d\vec{s}\,dt=I_{ijk}\]

The first constraint can be solved for the unknowns in $ijk$, $G_{ijk}$ and $\mathcal{R}_{ijk}$ (here assuming $D_{ijk}$ is neither 0 or its maximum value):
\[D_{ijk} \leq \mathcal{L}_0+\mathcal{L}_1 \cdot \left(\mathcal{R}_{ijk} + \mathcal{C}_{ij} \cdot \left(t_{k,1}-t_{k,0}\right) + \mathcal{E}\cdot G_{ijk}\right) < D_{ijk} + 1\]
\[\frac{D_{ijk}-\mathcal{L}_0}{\mathcal{L}_1} \leq\mathcal{R}_{ijk} + \mathcal{C}_{ij} \cdot \left(t_{k,1}-t_{k,0}\right) + \mathcal{E}\cdot G_{ijk} < \frac{D_{ijk} - \mathcal{L}_0}{\mathcal{L}_1} + \frac{1}{\mathcal{L}_1}\]
%\[\frac{D_{ijk}-\mathcal{L}_0}{\mathcal{E}\mathcal{L}_1} - \frac{\mathcal{R}_{ijk} + \mathcal{C}_{ij} \cdot (t_{k,1} - t_{k,0})}{\mathcal{E}} \leq G_{ijk} < \frac{D_{ijk} - \mathcal{L}_0}{\mathcal{E}\mathcal{L}_1} - \frac{\mathcal{R}_{ijk} + \mathcal{C}_{ij} \cdot (t_{k,1} - t_{k,0})}{\mathcal{E}} + \frac{1}{\mathcal{E}\mathcal{L}_1}\]
\[\frac{D_{ijk}-\mathcal{L}_0 - \mathcal{L}_1 \mathcal{C}_{ij} \cdot (t_{k,1} - t_{k,0})}{\mathcal{E}\mathcal{L}_1} \leq G_{ijk} + \frac{\mathcal{R}_{ijk}}{\mathcal{E}} < \frac{D_{ijk}-\mathcal{L}_0 - \mathcal{L}_1 \mathcal{C}_{ij} \cdot (t_{k,1} - t_{k,0})}{\mathcal{E}\mathcal{L}_1} + \frac{1}{\mathcal{E}\mathcal{L}_1}\]

\end{document}

